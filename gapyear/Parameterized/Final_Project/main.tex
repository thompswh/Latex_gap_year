% Page formatting
\documentclass[11pt]{article}
\usepackage[margin=1in]{geometry}

% Math packages
\usepackage{amsfonts, amsmath, amssymb, amsthm}
\usepackage{mathtools}

% Graphs
\usepackage{tikz}

% Formatting packages
\usepackage[inline,shortlabels]{enumitem}


% Color/box formatting
\usepackage[dvipsnames]{xcolor}
\usepackage[most]{tcolorbox}
\tcbuselibrary{theorems}
\tcbset{enhanced, breakable}
\usepackage{hyperref} % Links
\hypersetup{colorlinks=true, citecolor=blue, linkcolor = blue, urlcolor=blue}
\usepackage{titlesec} % Title
\usepackage{mdframed} % Boxes
\usepackage[twoside]{fancyhdr} % Header
\usepackage{mathtools}
\usepackage{enumitem}

% Algorithm / Code packages
\usepackage[linesnumbered,ruled,vlined]{algorithm2e}
\usepackage[sanserif,full]{complexity}


\setlength{\headheight}{15pt}
% -------------------------------------- TITLE ------------------------------------------
\title{Parameterized inapproximability: From PIH to Clique}
\author{Whitaker Thompson}
\date{\today}
% -------------------------------------- TITLE ------------------------------------------


% Commands
\newcommand{\CC}{\mathbb{C}}
\newcommand{\RR}{\mathbb{R}}
\newcommand{\QQ}{\mathbb{Q}}
\newcommand{\ZZ}{\mathbb{Z}}
\newcommand{\NN}{\mathbb{N}}
\newcommand{\FF}{\mathbb{F}}
\newcommand{\norm}[1]{\left\lVert #1 \right\rVert}
\newcommand{\curlyt}{\mathscr{T}}
\newcommand{\curlyc}{\mathscr{C}}
\newcommand{\curlyb}{\mathscr{B}}
\newcommand{\curlyu}{\mathscr{U}}
\newcommand{\curlyv}{\mathscr{V}}
\newcommand{\im}{\text{im}}
\newcommand{\tr}{\text{tr}}
\newcommand{\inv}[1]{#1^{-1}}
\DeclarePairedDelimiter{\abs}{\lvert}{\rvert}
\newcommand{\sol}{\textbf{Solution. }}
\newcommand{\negl}{\text{negl}}
\newcommand{\bit}{\{0,1\}}
\newcommand{\inner}[2]{\langle #1, #2 \rangle}
\newcommand{\Ex}{\mathbb{E}}
\newcommand{\argmin}{\text{argmin}}
\newcommand{\argmax}{\text{argmax}}



% Problems
\newcommand{\Clique}{\textsc{Clique}}
\newcommand{\VertexCover}{\textsc{VertexCover}}
\newcommand{\IndependentSet}{\textsc{IndependentSet}}
% Colored theorem environment

\newtcbtheorem[number within=subsection]
{definition}
{Definition}
{
    enhanced,              % allows advanced features
    breakable,             % allows page breaks
    colback=blue!5,        % background color
    frame hidden,
    coltitle=black,        % title text color
    colbacktitle=blue!15,  % title background
    fonttitle=\bfseries \large,   % title font
    separator sign={\;---\;}, % separator between title and body
    boxrule=0.8pt,         % frame thickness
    arc=0mm,               % rounded corners
    before skip=10pt,      % vertical space before box
    after skip=10pt,       % vertical space after box
  }
{def}

\newtcbtheorem[number within=subsection]
{ex}
{Example}
{
    enhanced,              % allows advanced features
    breakable,             % allows page breaks
    colback=ForestGreen!10,        % background color
    frame hidden,
    coltitle=black,        % title text color
    colbacktitle=ForestGreen!35,  % title background
    fonttitle=\bfseries \large,   % title font
    separator sign={\;---\;}, % separator between title and body
    boxrule=0.8pt,         % frame thickness
    arc=0mm,               % rounded corners
    before skip=10pt,      % vertical space before box
    after skip=10pt,       % vertical space after box
  }
{examp}


\newtcbtheorem[number within=subsection]
{thm}
{Theorem}
{
    enhanced,              % allows advanced features
    breakable,             % allows page breaks
    colback=BrickRed!5,        % background color
    frame hidden,
    coltitle=black,        % title text color
    colbacktitle=BrickRed!25,  % title background
    fonttitle=\bfseries \large,   % title font
    separator sign={\;---\;}, % separator between title and body
    boxrule=0.8pt,         % frame thickness
    arc=0mm,               % rounded corners
    before skip=10pt,      % vertical space before box
    after skip=10pt,       % vertical space after box
  }
{therm}


\newtcbtheorem[number within=subsection]
{note}
{Note}
{
    enhanced,              % allows advanced features
    breakable,             % allows page breaks
    colback=Gray!5,        % background color
    frame hidden,
    coltitle=black,        % title text color
    colbacktitle=Gray!20,  % title background
    fonttitle=\bfseries \large,   % title font
    separator sign={\;---\;}, % separator between title and body
    boxrule=0.8pt,         % frame thickness
    arc=0mm,               % rounded corners
    before skip=10pt,      % vertical space before box
    after skip=10pt,       % vertical space after box
  }
{therm}




\newtcbtheorem[number within=subsection]
{conjecture}
{Conjecture}
{
    enhanced,              % allows advanced features
    breakable,             % allows page breaks
    colback=Goldenrod!5,        % background color
    frame hidden,
    coltitle=black,        % title text color
    colbacktitle=Goldenrod!20,  % title background
    fonttitle=\bfseries \large,   % title font
    separator sign={\;---\;}, % separator between title and body
    boxrule=0.8pt,         % frame thickness
    arc=0mm,               % rounded corners
    before skip=10pt,      % vertical space before box
    after skip=10pt,       % vertical space after box
  }
{conje}




\newtcbtheorem[number within=subsection]
{problem}
{Problem}
{
  title=Title,
  colframe=black,
  colback=white,
  colbacktitle=white,
  coltitle=black,
  boxrule=0.8pt,
  arc=0pt
}
{prob}



\begin{document}

\pagestyle{fancy}
\fancyhf{}
%------------------------------------- HEADER --------------------------------------------
\fancyhead[R]{Whitaker Thompson}
\fancyhead[L]{Class}
%\fancyhead[C]{\today}
\fancyfoot[C]{\thepage}
%------------------------------------- HEADER --------------------------------------------

\maketitle
\thispagestyle{fancy} % For formatting the title page

We are reviewing \cite{CLIQUE-TO-PIH} in order to give a talk on it for my final pproject in the class I am fake auditing.

% -----------------------------COMMANDS FOR TOC ------------------------------------
\tableofcontents
\newpage


%---------------------------------- BEGIN -------------------------------------------
\section{Introduction}

There are strong inapproximability results for the $k$-clique problem. One gets straightaway from the PCP theorem and the FGLSS graph that there it is $\NP$-hard to approximate within a constant factor, and it turns out for any $\varepsilon > 0$, it is $\NP$-hard to approximate within a factor of $\frac{1}{n^{1-\varepsilon}}$. It is also known from the parameterization side that this problem is $\W[1]$-hard, and so an $\FPT$-algorithm would imply that $\P = \NP$, with $\W[1]$ playing the analogue role to $\NP$ in parameterized complexity. One desires to rule out approximation algorithms that are based on approximation \emph{and} parameterization, i.e.\ we want to be able to say that even in the parameterized setting, one cannot approximate $k$-$\Clique$ well.

It actually has been proven that a constant-factor approximation for $k$-$\Clique$is $\W[1]$-hard, as was shown in \cite{Constant-approximating-kclique-is-w[1]-hard}. This was quickly improved to show that it is $\W[1]$-hard to approximate within factor $k^{o(1)}$. Though they do not use a parameterized PCP theorem, the proof in \cite{Constant-approximating-kclique-is-w[1]-hard} follows the usual PCP-based approximation formula. That is, they do the following

\begin{enumerate}
	\item Reduce $k$-clique to an algebraic problem, $k$-vector sum

	\item Using Hadamard codes with local testing and decoding, $k$-vector sum is reduced to 2CSP

	\item The 2CSP instances are turned back into $k$-clique instances, but now with a gap

\end{enumerate}

It has been suggested that the constant inapproximability of the parameterized binary CSP would serve as a parameterized PCP theorem, and it has thus been dubbed the \emph{parameterized inapproximability hypothesis} (PIH henceforth.) The paper will serve as an explanation for these.

\section{Definitions}

We give a list of definitions here (only the unusual ones or ones I don't recognize):

\begin{definition}{Algebra}{algebra}

	Let $q$ be a prime power. Then $\FF_q$ is the unique finite field with $q$ elements. If $q$ itself is prime, then $\FF_q$ is a \emph{prime field}. Boldface letters $\textbf{u}$ denote vectors in $\FF_q^d$, and for every $v \in \FF_q^d$ and $i \in [d]$ is denoted as $v[i]$.
\end{definition}


\begin{definition}{FPT Reduction}{fptred}

	Let $P_1 = (Q_1, \kappa_1)$ and $P_2 = (Q_2, \kappa_2)$ be two parameterized problems. An $\FPT$-reduction from $P_1$ to $P_2$ is an $\FPT$-algorithm $R$ such that for every instance $x$ of $P_1$, the algorithm $R$ computes an instance $R(x)$ of $P_2$ satisfying \begin{itemize}[-]
		\item $x \in Q_1 \Longleftrightarrow R(x) \in Q_2$

		\item $\kappa_2(R(x)) \le g(\kappa_1(x))$ for a computable function $g : \NN \to \NN$ independent of $x$.

	\end{itemize}

	This means that $\W[1]$ is the set of all problems that can be reduced to $k$-$\Clique$ by an $\FPT$-reduction.
\end{definition}

\section{Parameterized optimization and approximation}

Instead of considering \textsc{Max-Clique} and the traditional $k$-clique problem, the paper will consider the multicolored $k$-clique problem:

\begin{problem}{Multi-colored $k$-clique}{mckc}

	Input: A graph $G$ such that $V(G) = V_1 \sqcup \cdots \sqcup V_k$ and each $V_i$ is an independent set.
	Parameter: $k$
	Output: Yes if $G$ contains a $k$-clique, no o.w.
	

\end{problem}

If $G$ is an instance of \ref{prob:mckc}, then $\omega(G)$ (independence number of $G$) is at most $k$, since each $V_i$ is an independent set. The reduction from $k$-clique is trivial; make $k$ disjoint copies of $V(G)$ and add an edge between $\{u,i\}$ and $\{v,j\}$ if $\{u,v\} \in E(G)$ and $i \ne j$. This new graph $G'$ has a $k$-clique if and only if $G$ does, so Max-Clique is $\FPT$ iff \ref{prob:mckc} is.

\newpage
\bibliographystyle{plain}
\bibliography{references}
\end{document}
