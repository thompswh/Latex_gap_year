% Page formatting
\documentclass[11pt]{article}
\usepackage[margin=1in]{geometry}

% Math packages
\usepackage{amsfonts, amsmath, amssymb, amsthm}
\usepackage{mathtools}
\usepackage{cite}
\bibliographystyle{plain}


% Graphs
\usepackage{tikz}

% Formatting packages
\usepackage{xcolor}
\usepackage[inline,shortlabels]{enumitem}
\usepackage{hyperref} % Links
\hypersetup{colorlinks=true, linkcolor = blue, urlcolor=blue, citecolor = blue}
\usepackage{titlesec} % Title
\usepackage{mdframed} % Boxes
\usepackage[twoside]{fancyhdr} % Header
\usepackage{mathtools}
\usepackage{enumitem}

% Algorithm / Code packages
\usepackage[linesnumbered,ruled,vlined]{algorithm2e}
\usepackage[sanserif,full]{complexity}



% -------------------------------------- TITLE ------------------------------------------
\title{Parameterized Inapproximability}
\author{Whitaker Thompson}
\date{\today}
% -------------------------------------- TITLE ------------------------------------------

% Format the section titles
\renewcommand{\thesection}{\Roman{section}}

% Commands
\newcommand{\CC}{\mathbb{C}}
\newcommand{\RR}{\mathbb{R}}
\newcommand{\QQ}{\mathbb{Q}}
\newcommand{\ZZ}{\mathbb{Z}}
\newcommand{\NN}{\mathbb{N}}
\newcommand{\FF}{\mathbb{F}}
\newcommand{\norm}[1]{\left\lVert #1 \right\rVert}
\newcommand{\curlyt}{\mathscr{T}}
\newcommand{\curlyc}{\mathscr{C}}
\newcommand{\curlyb}{\mathscr{B}}
\newcommand{\curlyu}{\mathscr{U}}
\newcommand{\curlyv}{\mathscr{V}}
\newcommand{\im}{\text{im}}
\newcommand{\tr}{\text{tr}}
\newcommand{\inv}[1]{#1^{-1}}
\DeclarePairedDelimiter{\abs}{\lvert}{\rvert}
\newcommand{\sol}{\textbf{Solution. }}
\newtheorem*{lemma}{Lemma}
\newtheorem{theorem}{Theorem}
\newtheorem{definition}{Definition}
\newtheorem{conjecture}{Conjecture}
\newcommand{\fpt}{\textsc{FPT}}

% EECS 475 Commands
\newcommand{\keyspace}{\ensuremath{\mathcal{K}}}
\newcommand{\msgspace}{\ensuremath{\mathcal{M}}}
\newcommand{\ctspace}{\ensuremath{\mathcal{C}}}
\newcommand{\tagspace}{\ensuremath{\mathcal{T}}}
\newcommand{\idspace}{\ensuremath{\mathcal{ID}}}
\newcommand{\skcgen}{{\sf Gen}} 
\newcommand{\skcenc}{{\sf Enc}}
\newcommand{\skcdec}{{\sf Dec}}
\newcommand{\negl}{\text{negl}}
\newcommand{\bit}{\{0,1\}}
\newcommand{\calD}{\mathcal{D}}
\newcommand{\inner}[2]{\langle #1, #2 \rangle}
\newcommand{\advan}{\textbf{Adv}}
\newcommand{\Ex}{\mathbb{E}}
\newcommand{\argmin}{\text{argmin}}
\newcommand{\argmax}{\text{argmax}}



\begin{document}

\pagestyle{fancy}
\fancyhf{}
%------------------------------------- HEADER --------------------------------------------
\fancyhead[R]{Whitaker Thompson}
\fancyhead[L]{Class}
%\fancyhead[C]{\today}
\fancyfoot[C]{\thepage}
%------------------------------------- HEADER --------------------------------------------

\maketitle
\thispagestyle{fancy} % For formatting the title page


% -----------------------------COMMANDS FOR TOC ------------------------------------
%\tableofcontents
%\newpage


%---------------------------------- BEGIN -------------------------------------------

Clique is inapproximable due to the PCP Theorem, and we know that Clique is $W[1]$-hard, but can we combine these approaches to get a FPT approximation algorithm? We introduce some important conjectures in the area:

\begin{conjecture}[ETH] \label{ETH}
    There is no $2^{o(n)}$ time algorithm for solving $\textsc{3SAT}$
\end{conjecture}

This is only for exact versions, so we also introduce the stronger statement that it is hard to even approximate $\textsc{3SAT}$:

\begin{conjecture}[Gap-ETH] \label{GAP-ETH}
    Given a $\textsc{3SAT}$ instance $\phi$ and any $\varepsilon > 0$, the $(1- \varepsilon, \varepsilon)$-$\textsc{3SAT}$ problem is $\NP$-hard.
\end{conjecture}

\cite{GAP} suggests that unless \ref{GAP-ETH} fails, there is no $\omega(k)$ fpt-approximation for the $k$-clique problem.

\section{Notation}

\begin{itemize}
    \item $\binom{S}{k}$ is the set of all $k$-subsets of $S$

    \item $\omega(G)$ is the size of a maximum clique in $G$

    \item Parameterized reduction -- Let $(Q_1, \kappa_1), (Q_2, \kappa_2)$ be two parameterized problems. An fpt-reduction from $(Q_1, \kappa_1)$ to $(Q_2, \kappa_2)$ is an fpt-algorithm $R$ such that for every instance $x$ of $Q_1$ the algorithm $R$ computes an instances $R(x)$ of $Q_2$ satisfying \begin{itemize}
        \item[-] $x \in Q_1 \Longleftrightarrow R(x) \in Q_2$

        \item[-] $\kappa_2(R(x)) \le g(\kappa_1(x))$ for a computable function $g : \NN \to \NN$ (independent of $x$).

        \end{itemize}
    \item $W[1]$ consists of all parameterized problems that can be reduced to $k$-Clique by an fpt-reduction.

    \item \begin{definition} \label{max-clique-def}
        Let $\rho: \NN \to \NN$. Then an algorithm $\mathcal{A}$ is an fpt-approximation of \textsc{Max-Clique} of ratio $\rho$ if \begin{itemize} 
        \item On an input graph $G$ the algorithm $A$ a computes a clique in $G$ of size at least 
        \[\frac{\omega(G)}{\rho(\omega(G))}\]

        \item the running time of $\mathcal{A}$ can be bounded by $f(\omega(W))n^{O(1)}$ for some computable $f: \NN \to \NN$
        \end{itemize}
    \end{definition}
\end{itemize}

\section{Parameterized optimization and approximation}

Based on \ref{max-clique-def}, there is always a trivial $k$ approximation (output a single vertex), and this can be improved to $\lceil k/c \rceil$ for any constant $c \ge 1$ (naively enumerate all $\binom{n}{c} = O(n^c)$ subsets and check if they are cliques, but the authors conjecture that this is the best they can do:

\begin{conjecture}
    For every $p : \NN \to \NN$ such that $\rho(k) = o(k)$, there is no $\rho$-approximation for \textsc{Max-Clique}.
\end{conjecture}

\section{Constraint Satisfaction Problem}

The parameterized inapproximability hypothesis states that $\textsc{Max2CSP}$ is constant-inapproximable by an $\fpt$ algorithm:

\begin{conjecture}[Parameterized Inapproximability Hypothesis] \label{PIH}
        There is a constant $0 < \varepsilon < 1$ such that it is $W[1]$-hard to distinguish between \begin{itemize}
            \item[-] Satisfiable \textsc{2CSP} instances

            \item[-] \textsc{2CSP} instances where any assignment cannot satisfy $\varepsilon$-fraction of the constraints
        \end{itemize}
\end{conjecture}

This amounts to showing that there is an \fpt-reduction $R$ from $k$-clique to $\textsc{Max2CSP}$ with the following properties: \begin{itemize}
    \item[-] If $G$ has a $k$-clique, then $R(G)$ is satisfiable

    \item[-] If $G$ has no $k$-clique, then no assignment satisfies an $\varepsilon$ fraction of the constraints of $R(G)$
\end{itemize}

The important result is that assuming \ref{PIH}, it is $W[1]$-hard to approximate $k$-clique within a constant factor, with the analysis following the FLGSS graph \cite{FLGSS}. Also, we note that even though \ref{PIH} may be for a specific $\varepsilon'$, we can use parallel repetition to achieve inapproximability for \emph{any} $\varepsilon > 0$, which implies for any $c > 1$, there is no $c$-approximation.

\section{Coding Theory}

Fix a finite field $\mathbb{F}$. Note: This section is confusing, and to me (someone who is not studied in coding theory), they introduce a parallelized hadamard code which is in reality just matrix multiplication? Idk tho I will come back to this one

\begin{definition}[Sidon Sets] \label{sidon}
    Let $\FF$ be a finite field and $d \ge 1$. A subset $S \subseteq \FF^d$ is a \emph{linear Sidon Set} if for all $r, r' \in \FF*$ and $u, u', v, v' \in S$ with $u \ne u'$ and $v \ne v'$, we have
    \[ru + r'u' = rv + r'v' \Rightarrow \{u,u'\} = \{v,v'\}\]
\end{definition}

With some lemma for finding Sidon sets, we assume that for an instance of $k$-clique, $V(G)$ can be represented by a linear Sidon set.

\subsection{A simple combinatorial proof for the super constant inapproximability of clique}

There are proofs of lower bounds of $k^{o(1)}$-ratio inapproximability of $k$-clique that rely on various coding theory techniques, but recently there was a purely combinatorial proof \cite{chen2024simplecombinatorialconstructionko1lower} that does not rely on coding theory and the authors will present it here as a 2-step reduction, with an intermediary stop at a \textsc{CSP}.

\subsubsection{A forgotten definition}

We introduce something that I should've introduced earlier, a way to parameterize the clique problem by something that is not $\omega(G)$; consider the multi-colored $k$-clique problem:

\begin{definition}[$\textsc{MC}$-$k$-$\textsc{Clique}$] \label{mckclique}
    Instance: A graph $G$ and $k \ge 1$ such that $V(G) = \bigsqcup_{i \in [k]} V_i$, where each $V_i$ is an independent set.

    Parameter: $k$

    Problem: Decide whether $G$ has a $k$-clique.
\end{definition}

The \fpt-reduction from $k$-clique to \ref{mckclique} is trivial, i.e. make $k$ disjoint copies of $V(G)$ and add an edge between $(u,i)$ and $(v,j)$ if $\{u,v\} \in E(G)$ and $i \ne j$, and clearly the original has a $k$-clique if and only if the multi-colored version does too. 

\subsubsection{$k$-clique to \textsc{CSP}}

Assume that we are dealing with the multi-colored version, and the CSP variable set will be the following:
\[X \coloneqq \{ x_{\overline{r}}\, | \,\overline{r} = (r_1, \ldots, r_k) \in \FF^k \}\]
where each variable is supposed to take the value 
\[x_{\overline{r}} \coloneqq \mathcal{H}^d_k(v_1, \ldots, v_k)(\overline{r}) = r_1v_1 + \cdots + r_kv_k \in \FF^d\]

for a purported clique on vertices $v_1 \in V_1, \ldots, v_k \in V_k$. NOTE: This means that each $v \in V$ is represented by some $y \in \FF^d$

To check whether an assignment really gives us a clique in $G$, we have the following tests as constraints:\begin{itemize} 
    \item[-] \textbf{Vertex Test}: for every $i \in [k], \overline{r} \in \FF^k, r \in \FF^*$, test whether 
    \[x_{\overline{r} + re_i} - x_{\overline{r}} \in rV_i\]
    where $rV_i \coloneqq \{ra\, | \, a \in V_i\}$.

    \item[-] \textbf{Edge Test}: For every $\overline{r} = (r_1, \ldots, r_k) \in \FF^k, 1 \le i < i' \le k$, and $r, r' \in \FF^*$, test whether 
    \[x_{\overline{r} + re_i + r'e_{i'}} - x_{\overline{r}} = rv + rv'\]
\end{itemize}

So far, the \textsc{CSP} instance has the following properties:
\begin{itemize}
    \item Variables: $X  = \{x_{\overline{r}}\, |\, (r_1, \ldots, r_k) \in \FF^k\}$

    \item Values: For each $x \in \FF^k$, it can take a value $y \in \FF^d$

    \item Constraints: \begin{itemize}
		    \item Vertex Test: For every $i \in [k]$, $\overline{r} \in \FF^k$, and $r \in \FF^*$, test whether 
			    \[ x_{\overline{r} + re_i} - x_{\overline{r}} \in rV_i\]
			    
		where $rV_i \coloneqq \{ra \ | \ a \in V_i\}$ 

    \end{itemize}




\end{itemize}

\bibliography{references}

\end{document}

