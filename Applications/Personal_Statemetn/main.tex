% Page formatting
\documentclass[11pt]{article}
\usepackage[margin=1in]{geometry}

% Math packages
\usepackage{amsfonts, amsmath, amssymb, amsthm}
\usepackage{mathtools}

% Graphs
\usepackage{tikz}

% Formatting packages
\usepackage{xcolor}
\usepackage[inline,shortlabels]{enumitem}
\usepackage{hyperref} % Links
\hypersetup{colorlinks=true, linkcolor = navyblue, urlcolor=blue}
\usepackage{titlesec} % Title
\usepackage{mdframed} % Boxes
\usepackage[twoside]{fancyhdr} % Header
\usepackage{mathtools}
\usepackage{enumitem}

% Algorithm / Code packages
\usepackage[linesnumbered,ruled,vlined]{algorithm2e}
\usepackage[sanserif,full]{complexity}



% -------------------------------------- TITLE ------------------------------------------
\title{Shitty First Draft}
\author{Whitaker Thompson}
\date{\today}
\setlength{\headheight}{15pt}
% -------------------------------------- TITLE ------------------------------------------

% Format the section titles
\renewcommand{\thesection}{\Roman{section}}

% Commands
\newcommand{\CC}{\mathbb{C}}
\newcommand{\RR}{\mathbb{R}}
\newcommand{\QQ}{\mathbb{Q}}
\newcommand{\ZZ}{\mathbb{Z}}
\newcommand{\NN}{\mathbb{N}}
\newcommand{\FF}{\mathbb{F}}
\newcommand{\norm}[1]{\left\lVert #1 \right\rVert}
\newcommand{\curlyt}{\mathscr{T}}
\newcommand{\curlyc}{\mathscr{C}}
\newcommand{\curlyb}{\mathscr{B}}
\newcommand{\curlyu}{\mathscr{U}}
\newcommand{\curlyv}{\mathscr{V}}
\newcommand{\im}{\text{im}}
\newcommand{\tr}{\text{tr}}
\newcommand{\inv}[1]{#1^{-1}}
\DeclarePairedDelimiter{\abs}{\lvert}{\rvert}
\newcommand{\sol}{\textbf{Solution. }}
\newtheorem*{lemma}{Lemma}

% EECS 475 Commands
\newcommand{\keyspace}{\ensuremath{\mathcal{K}}}
\newcommand{\msgspace}{\ensuremath{\mathcal{M}}}
\newcommand{\ctspace}{\ensuremath{\mathcal{C}}}
\newcommand{\tagspace}{\ensuremath{\mathcal{T}}}
\newcommand{\idspace}{\ensuremath{\mathcal{ID}}}
\newcommand{\skcgen}{{\sf Gen}} 
\newcommand{\skcenc}{{\sf Enc}}
\newcommand{\skcdec}{{\sf Dec}}
\newcommand{\negl}{\text{negl}}
\newcommand{\bit}{\{0,1\}}
\newcommand{\calD}{\mathcal{D}}
\newcommand{\inner}[2]{\langle #1, #2 \rangle}
\newcommand{\advan}{\textbf{Adv}}
\newcommand{\Ex}{\mathbb{E}}
\newcommand{\argmin}{\text{argmin}}
\newcommand{\argmax}{\text{argmax}}



\begin{document}

\pagestyle{fancy}
\fancyhf{}
%------------------------------------- HEADER --------------------------------------------
\fancyhead[R]{Whitaker Thompson}
\fancyhead[L]{Class}
%\fancyhead[C]{\today}
\fancyfoot[C]{\thepage}
%------------------------------------- HEADER --------------------------------------------

\maketitle
\thispagestyle{fancy} % For formatting the title page


% -----------------------------COMMANDS FOR TOC ------------------------------------
%\tableofcontents
%\newpage


%---------------------------------- BEGIN -------------------------------------------
I have always loved math, and I didn't start coding until I got to college. I only took a coding class because it was required for the math major. I enjoyed the coding so much that I added the computer science major as well. I really enjoyed the discrete math classes, but found myself most excited by the required computer science theory course during my sophomore year. The following semester, I tried CS theory but because of some personal issues, mistakenly believed that I didn't really want to pursue it, so the next semester, I did not take any CS theory courses. However, in my free time (indeed, in the time I should've spent working on my other classes), I found looking into TCS on my own, and I spent the following summer doing research, and I was hooked. I spent my last year of undergrad taking as many TCS courses as I could. I am fascinated by the existence of $\NP$-complete languages, and I find that I (along with all computer scientists) are most motivated by that question, and so I want to spend my life working toward that problem. At the current historical moment, what that looks like is to work on relaxations of that problem; can we \emph{approximate} these problems well, can we come up with algorithms that are maybe polynomial in the entire input size while allowing for some superpolynomial factors in other parameters of the problem? Can these approaches be combined? Under the $\P \ne \NP$ assumption, can we prove that versions of this are impossible?

I spent much of my

\newpage

What I want my personal statement to convey:\begin{itemize}
	\item[-]  Euiwoong once told me that in addition to having the academic ability for PhD, schools also
look for students to have the personality for a PhD, and I need to convey that I have that.
What kind of personality are they likely looking for?\begin{itemize}
	\item[-]  Curiosity

	\item[-]  Commitment to working on a long-term problem
\end{itemize}
I just need to convey that I have these things


\item[-] I also made my decision to stay in the area, working at Kroger in order to continue understanding TCS. I have to be able to convey this was intentional, even though it sort of appears that there was no other option. One way to convey this is to state that I didn’t even consideranother option, which would show my dedication (although certainly this is less impressive than having another option and declining it.)
 
\item[-]I don’t know if this is actually required, but the reader should know that I genuinely love thinking about theory, that I really feel that the problems are important, and that I like thinking about them.

\item[-] I have been a benefactor of Michigan’s burgeoning theory department (and I have to use
that word since it sounds smart and Jerry Seinfeld used it once), and I want to contribute
to a robust theory department. I want to make it a place where graduate students can
thrive, professionally and personally, and where TCS is not seen as a chore to get through
but that it is a topic worth thinking about that makes better people better mathematicians,
programmers, and contributing members of society.

\end{itemize}

\newpage


I took my first theory of computation course during my sophomore year, and I was obsessed with the Cook-Levin Theorem and the existence of natural $\NP$-complete problems. I was disappointed to hear that progress on the conjecture has been relatively slow, but this disappointment was short lived when I learned that there have been many meaningful, beautiful results to certain relaxations of that problem. I took an algorithms course the following semester that was taught my Euiwoong Lee, and quickly became entranced by approximation algorithms. 

At the end of the semester, however, with my junior year halfway done and still no internship prospects on the horizon, I decided to take some courses that would boost my software engineering resume, and did not sign up for any theory courses. However, I found that in all of my spare time that following semester, I was reading about algorithms and complexity, and instead of looking for an intership that semester, I reached out to Euiwoong Lee and asked if I could work on a research problem with him. He said he would be happy to work with me, and in the words of Chris Peikert, who taught my introductory theory of computation course, I ``joined the big-O life for good." 

I spent my last year of undergrad taking as many theory courses as I could, and I came to appreciate the effect that a robust theory department has on it's students, graduate and undergraduate alike. I was around graduate students and professors who were among the most proficient in their field, and their curiosity and love of the material imbued a similar enthusiasm in me. I look forward to contributing to a theory department not just as a student and a researcher, but also as a member of the culture of that department.

For the 2025-2026 school year, I accepted a job as a supplemental math tutor at a high school in Boston through Americorps. Unfortunately, in late June, I found out that due to funding issues, the job would no longer be offered. In order to continue developing as a computer scientist, I opted to stay in Ann Arbor, working a retail job while I unofficially audited some graduate theory courses at Michigan. This has been a fantastic way to stay connected to a strong department, and to grow intellectually while financially supporting myself. 
\newpage

\begin{itemize}[-]
	\item Add more variance in sentence length

	\item Second to last paragraph is non-specific; talk about something specific about Michigan theory's department that inspired you (personal statement, i.e. anecdotes are in your favor)

	\item "At the end of the semester, ..." is long, and talking about lack of prospects is kind of negative, so maybe talk about it more like you tried software engineering and didn't like it or that you liked theory \emph{more}

	\item For the conclusion \begin{itemize}
			\item What have you learned in the last couple months?

			\item What am I looking forward to in my educational journey?

			\item If only for 1 school, insert interest in [SCHOOL] here

	\end{itemize}

\end{itemize}



\end{document}
