
% Page formatting
\documentclass[11pt]{article}
\usepackage[margin=1in]{geometry}

% Math packages
\usepackage{amsfonts, amsmath, amssymb, amsthm}
\usepackage{mathtools}

% Graphs
\usepackage{tikz}

% Formatting packages
\usepackage{xcolor}
\usepackage[inline,shortlabels]{enumitem}
\usepackage{hyperref} % Links
\hypersetup{colorlinks=true, linkcolor = navyblue, urlcolor=blue}
\usepackage{titlesec} % Title
\usepackage{mdframed} % Boxes
\usepackage[twoside]{fancyhdr} % Header
\usepackage{mathtools}
\usepackage{enumitem}

% Algorithm / Code packages
\usepackage[linesnumbered,ruled,vlined]{algorithm2e}
\usepackage[sanserif,full]{complexity}


\setlength{\headheight}{15pt}

% -------------------------------------- TITLE ------------------------------------------
\title{Statement of Purpose}
\author{Whitaker Thompson}
\date{\today}
% -------------------------------------- TITLE ------------------------------------------

% Format the section titles
\renewcommand{\thesection}{\Roman{section}}

% Commands
\newcommand{\CC}{\mathbb{C}}
\newcommand{\RR}{\mathbb{R}}
\newcommand{\QQ}{\mathbb{Q}}
\newcommand{\ZZ}{\mathbb{Z}}
\newcommand{\NN}{\mathbb{N}}
\newcommand{\norm}[1]{\left\lVert #1 \right\rVert}
\newcommand{\curlyt}{\mathscr{T}}
\newcommand{\curlyc}{\mathscr{C}}
\newcommand{\curlyb}{\mathscr{B}}
\newcommand{\curlyu}{\mathscr{U}}
\newcommand{\curlyv}{\mathscr{V}}
\newcommand{\im}{\text{im}}
\newcommand{\tr}{\text{tr}}
\newcommand{\inv}[1]{#1^{-1}}
\DeclarePairedDelimiter{\abs}{\lvert}{\rvert}
\newcommand{\sol}{\textbf{Solution. }}
\newtheorem*{lemma}{Lemma}

% EECS 475 Commands
\newcommand{\keyspace}{\ensuremath{\mathcal{K}}}
\newcommand{\msgspace}{\ensuremath{\mathcal{M}}}
\newcommand{\ctspace}{\ensuremath{\mathcal{C}}}
\newcommand{\tagspace}{\ensuremath{\mathcal{T}}}
\newcommand{\idspace}{\ensuremath{\mathcal{ID}}}
\newcommand{\skcgen}{{\sf Gen}} 
\newcommand{\skcenc}{{\sf Enc}}
\newcommand{\skcdec}{{\sf Dec}}
\newcommand{\negl}{\text{negl}}
\newcommand{\bit}{\{0,1\}}
\newcommand{\calD}{\mathcal{D}}
\newcommand{\inner}[2]{\langle #1, #2 \rangle}
\newcommand{\advan}{\textbf{Adv}}
\newcommand{\Ex}{\mathbb{E}}
\newcommand{\argmin}{\text{argmin}}
\newcommand{\argmax}{\text{argmax}}



\begin{document}

\begin{centering}

{\LARGE Statement of Purpose}

\vspace*{2mm}
Whitaker Thompson 

\end{centering}

\noindent\rule{\textwidth}{0.4pt}


I hope to pursue a Ph.D. in theoretical computer science to study discrete optimization, particularly approximation algorithms and hardness of approximation for $\NP$-hard problems. I also enjoy thinking about the interaction of these areas with parameterized algorithms, and establishing lower bound and hardness results for the approximate and parameterized versions of these problems.

With Euiwoong Lee, I studied the Unique Games problem when the constraint graph is a metric graph, i.e.\ a complete graph equipped with a weight that obeys the triangle inequality. The Unique Games problem is central in the theory of hardness of approximation; it is conjectured to be $\NP$-hard to distinguish between almost and barely satisfiable cases. This conjecture implies the optimality of approximation algorithms for several canonical $\NP$-hard problems, including \textsc{VertexCover} and \textsc{MaxCut}. Understanding the approximability of the problem on more restricted instances such as ours can give insight into the general case. 

An existing algorithm for the Unique Games problem rounds a solution to a semidefinite programming relaxation of the problem. The quality of the approximation is then dependent on the expansion of the constraint graph. Our initial approach was to try and prove that metric graphs expand well, i.e.\ that the second largest eigenvalue of the normalized adjacency matrix of the graph is bounded above by some constant. I was able to prove a weaker version of this fact, but ultimately switched directions.

In similar problems, where the goal is to assign some labels to vertices to satisfy constraints along edges, an algorithmic paradigm that has been successful is that of using a \emph{pivot}. In these algorithms, a vertex is designated as the pivot, and the label for every vertex in the neighborhood of the pivot is chosen with respect to the constraint along the edge that is shared with the pivot. These algorithms are beautifully simple, and while the choices they make are local, they actually end up leading to good global assignments. This approach has been successful for Unique Games on unweighted complete graphs. To generalize this to metric weighted graphs, we came up with a randomized algorithm that selects a pivot vertex and then rounds a solution to an extended linear programming relaxation of the problem. Our analysis crucially used the triangle inequality and resulted in the first constant-factor approximation algorithm for our problem. This project culminated in my senior honors thesis, which I presented just before my graduation in April 2025.

Though our result was substantial, the approximability of our problem was not entirely resolved yet. After my graduation, I spent time by myself trying to find a polynomial-time approximation scheme (PTAS) for the problem. I reviewed PTASes for metric versions of graph problems that used existing PTASes on dense versions as a subroutine, and this is where I became acquainted with and interested in the approximability of CSPs. I was drawn to them for their generality, and I continue to actively work toward using existing algorithms for them to solve my problem.

Going into my final semester of undergraduate studies, I was completely set on pursuing a career in theoretical computer science research and decided to apply to graduate school during the following cycle. For the 2025-2026 school year, I accepted a job through Americorps as a math tutor at a high school in Boston for students who needed supplemental instruction. Unfortunately, in late June, I found out that due to funding issues, the job would not be offered for the upcoming school year. I was working a retail job in Ann Arbor for the summer in order to save money for graduate school applications, and I decided to continue working at that job throughout the 2025-2026 school year, which would allow me to support myself financially and to audit some graduate theoretical computer science courses that were being offered at Michigan in the fall of 2025. This has been a great way to continue thinking about algorithms and familiarizing myself with different areas of theory.

One of the courses I have been auditing this year is a special topics class on parameterized and fast exponential algorithms. I previously had seen parameterized algorithms in a graph algorithms course I took during my final undergraduate semester, and the course I am auditing is being taught by a professor who mainly works in approximation algorithms, so there was more of a focus on the interaction of parameterized algorithms with approximation. I have become interested in whether or not using both parameterization \emph{and} approximation can give an advantage for problems that each individual subarea suggests is hard. In pursuit of this, for my final project in the class, I am preparing a talk on a recent paper that surveys the parameterized approximability of the clique problem in order to acquaint myself with the literature.

I would very excited to work with Debmalya Panigrahi. I was especially fond of Professor Panigrahi's recent paper that examined the parameterized complexity of local search for CSPs. A question they proposed that interested me was whether or not CSPs on structured instances could be usefully parameterized. In trying to find a PTAS for my research problem, I studied the PTASes on these structured CSPs instances, and I would like to use the intuition I developed to examine this problem.

Studying parameterization and its interaction with approximation has also made me think about approximation within non-traditional computational models. Professor Panigrahi's recent papers on CSPs with oracle advice and maintaining an approximately optimal dynamic set cover tell me that this is also an area that he often works in, and working with him would be a great way to explore these types of problems.


\end{document}
